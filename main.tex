\documentclass[10pt, xcolor = svgnames]{beamer} %Beamer
\usepackage{palatino} %font type
\usefonttheme{metropolis} %Type of slides
\usefonttheme[onlymath]{serif} %font type Mathematical expressions
\usetheme[progressbar = frametitle,titleformat frame=smallcaps,numbering=counter]{metropolis} %This adds a bar at the beginning of each section.
\useoutertheme[subsection=false]{miniframes} %Circles in the top of each frame, showing the slide of each section you are at

\usepackage{appendixnumberbeamer} %enumerate each slide without counting the appendix
\setbeamercolor{progress bar}{fg=Maroon!70!Coral} %These are the colours of the progress bar. Notice that the names used are the svgnames
\setbeamercolor{title separator}{fg=DarkSalmon} %This is the line colour in the title slide
\setbeamercolor{structure}{fg=black} %Colour of the text of structure, numbers, items, blah. Not the big text.
\setbeamercolor{normal text}{fg=black!87} %Colour of normal text
\setbeamercolor{alerted text}{fg=DarkRed!60!Gainsboro} %Color of the alert box
\setbeamercolor{example text}{fg=Maroon!70!Coral} %Colour of the Example block text


\setbeamercolor{palette primary}{bg=NavyBlue!50!DarkOliveGreen, fg=white} %These are the colours of the background. Being this the main combination and so one. 
\setbeamercolor{palette secondary}{bg=NavyBlue!50!DarkOliveGreen, fg=white}
\setbeamercolor{palette tertiary}{bg=NavyBlue!40!Black, fg= white}
\setbeamercolor{section in toc}{fg=NavyBlue!40!Black} %Color of the text in the table of contents (toc)

%These next packages are the useful for Physics in general, you can add the extras here. 
\usepackage{amsmath, amssymb}
\usepackage{slashed}
\usepackage{cite}
\usepackage{relsize}
\usepackage{caption}
\usepackage{subcaption}
\usepackage{multicol}
\usepackage{booktabs}
\usepackage[scale=2]{ccicons}
\usepackage{pgfplots}
\usepgfplotslibrary{dateplot}
\usepackage{geometry}
\usepackage{xspace}
\usepackage[useregional]{datetime2}
\usepackage{framed}
\usepackage[strict]{changepage}



\newcommand{\themename}{\textbf{\textsc{bluetemp}\xspace}}%metropolis}}\xspace}




% environment derived from framed.sty: see leftbar environment definition
\definecolor{formalshade}{rgb}{0.95,0.95,1}
\definecolor{mygray}{gray}{0.4}
\definecolor{lightgray}{gray}{0.93}

\newenvironment{formal}{%
  \def\FrameCommand{%
    \hspace{1cm}%  ← change this to modify the indentation ========================
    {\color{mygray}\vrule width 2pt}%
    {\color{lightgray}\vrule width 4pt}%
    \colorbox{lightgray}%
  }%
  \MakeFramed{\advance\hsize-\width\FrameRestore}%
  \noindent\hspace{-4.55pt}% disable indenting first paragraph
  \begin{adjustwidth}{}{7pt}%
  \vspace{2pt}\vspace{2pt}%
}
{%
  \vspace{2pt}\end{adjustwidth}\endMakeFramed%
}


\title{Your amazing Title}
\author[Name]{Your name\inst{$\dagger$}, others\inst{$\ddagger$} and supervisor's name\inst{$\ast$}} %With inst, you can change the institution they belong
\subtitle{Subtitle}


\institute[uni] % (optional)
{
	\inst{\dagger}
	Department of Physics \\
	\textsc{National Taiwan University}
	\and
	\inst{\ddagger}
	Department of Something \\
	\textsc{others' Institute}
	\and
	\inst{\ast}
	Department of Something \\
	\textsc{Supervisor's Institute}
}


% \date{\today} %Here you can change the date
\date{\DTMdate{2023-12-31}} %Here you can change the date

\titlegraphic{\vspace{0.0cm}\hfill\includegraphics[width = 0.25\textwidth]{logo.pdf}}

\begin{document}
{
\setbeamercolor{background canvas}{bg=NavyBlue!50!DarkOliveGreen, fg=white}
\setbeamercolor{normal text}{fg=white}
\maketitle
}%This is the colour of the first slide. bg= background and fg=foreground

\metroset{titleformat frame=smallcaps} %This changes the titles for small caps




\begin{frame}{Outline}
  \setbeamertemplate{section in toc}[sections numbered] %This is numbering the sections
  \tableofcontents[hideallsubsections] %You can comment this line if you want to show the subsections in the table of contents
\end{frame}




\begin{frame}{Objectives}
\underline{\textsc{Some text:}}
\begin{small}
This is some small Text. 
\end{small}

\metroset{block=fill}
\begin{exampleblock}{\textsc{Example block}}
\begin{itemize}
    \item You know how to do itemize
    \item Also here
\end{itemize}
\end{exampleblock}
\end{frame}



\section{Introduction}



\begin{frame}[fragile]{Introduction: blah blah} %You can change fragile by standout
Text Text Text Text. \\You can change the size of the footnote text like  Text\footnote{\small{ here.}} Text\footnote{\large{And here.}} Text\footnote{\tiny{And here.}}
\begin{itemize} %The symbol of the items can be changed by which ever you want, this is just an example.
    \item[$\diamond$] Text,
    \item[$\diamond$] Text,
    \item[$\diamond$] Text.
\end{itemize}
An equation without number could be represented by:
\begin{equation*}
    c^{2} = a^{2} + b^{2}
\end{equation*}
That's all for this slide.
\end{frame}



\begin{frame}[fragile]{Introduction: blah blah}
\begin{itemize}
    \item \textbf{Energy dissipation} (excluding frictional heat): assumed to occur near the rupture tip.
    \item \textbf{Key Question:} Does dissipation localize in natural faults despite complexities (e.g., inelasticity, off-fault damage)?
    \item \textbf{Implications:} Applicability of LEFM depends on maintaining a separation of scales.
\end{itemize}


\begin{formal}
	A key question is if energy dissipation (aside from frictional heat) on natural faults with all its complexities (e.g., weakening, off-fault inelasticity) truly is localized in the vicinity of the rupture tip.
\end{formal}
\end{frame}



\begin{frame}[standout]{This is other type of slide}
There is some text here.
And an equation with number:
\begin{equation}
    E^{2} = m^{2} + p^{2}
\end{equation}
\end{frame}


\begin{frame}{Include a Figure}
\begin{small}
Something about the fig.
\end{small}
\begin{figure}
\centering
\includegraphics[width=0.7\textwidth]{./fig/fig1}
\caption{Screenshot from the IMDb website.}
\label{fig1}
\end{figure}
\end{frame}


\begin{frame}[fragile]{Another way to include a Figure}
\begin{minipage}{0.45\textwidth}
\begin{enumerate}
	\item Released in 2014
	\item A comedy-drama
	\item Directed by Wes Anderson
\end{enumerate}
\end{minipage}
\hspace{20pt}
\begin{minipage}{0.45\textwidth}
\begin{figure}
\centering
\includegraphics[width = 0.8\textwidth]{./fig/poster.jpg}
\caption{Poster.}
\label{fig1}
\end{figure}
\end{minipage}
\end{frame}



\include{Othersection}



\section{This is another section}
\begin{frame}{Frame Title} %You can also not write fragile or standout and you can see how it looks
    Hello world!
\end{frame}



\section{Final section}



\begin{frame}{Conclusion}
    These are the final words, you do your best to try to wake up everyone that was listening to your talk.
\end{frame}


{\setbeamercolor{palette primary}{fg=black, bg=orange!30} %You can change the colours
\begin{frame}[standout]
  Thank you! And thank to yourself because you did all the job. 
\end{frame}
}
%\appendix
%
%\begin{frame}{Back up}
%    These slides won't appear in the table of contents and will not be counted as the total slides.
%\end{frame}

% Appendix will cause error
\section{Bibliography}

\begin{frame}[allowframebreaks]{Bibliography}
\nocite{*}
\bibliographystyle{unsrt}
\bibliography{./bibliography.bib}
\end{frame}


\end{document}












